\section{Выводы}
Выполнив Курсовой проект по курсу \enquote{Дискретный анализ}, я познакомился с такой тяжелой, но очень интересной темой, как сжатие данных. Данное задание заставило меня вспомнить почти все, что я прошел за 2 семестра изучения дисциплины: деревья, очереди, суффиксные массивы, динамическое программирование, алгоритмы сжатия, сортировки за линейное время - все это мне пришлось использовать в этой работе. Также не обошлось без знаний ООП для реализации наследуемых классов и виртуальных функций.\\

Из минусов своей реализации я вынужден отметить то, что я применяю последоательно алгоритмы сжатия для временных файлов побуфферно, что тратит отгромное количество памяти на жестком диске и занимает время на перезаписывание. Мне стоило считывать кодируемы файл побуфферно, после чего последоательно применять алгоритмы сатия непосредстенно к буфферу.\\

Однако несмотря на всю несоершенность моей реализации я получил огромный опыт при работе над курсовым проектом и, возможно, буду использоть его в дальнейшем в качестве портфолио и демонстрции своих навыков.
\pagebreak